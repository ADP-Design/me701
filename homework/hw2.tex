\documentclass[11pt]{article}
\usepackage{mathtools,hyperref,booktabs,fullpage, txfonts}
\usepackage[amssymb,cdot]{SIunits}
\usepackage[utopia]{mathdesign}     

\usepackage[table]{xcolor}
\usepackage{amsmath}
\usepackage{hyperref}
\usepackage{longtable}
\usepackage{fullpage}
 
\definecolor{lightgray}{gray}{0.93}

\pagestyle{empty}
\setlength\parindent{0pt}
\renewcommand{\thefootnote}{\fnsymbol{footnote}}
 
\makeatletter
\renewcommand\section{\@startsection{section}{1}{\z@}%
                                  {-3.5ex \@plus -1ex \@minus -.2ex}%
                                  {2.3ex \@plus.2ex}%
                                  {\normalfont\bfseries}}
\makeatother


\begin{document}

{\large
  \begin{center}
    {\bf ME 701 -- Development of Computer Applications In Mechanical Engineering \\ 
         Homework 2 -- Due 9/13/2017}         
    }
  \end{center}
}
 

\section*{Problem 1 -- More in the Command Line}

\begin{enumerate}
\item Create two files, {\tt a.txt} and {\tt b.txt}, that have your 
      first and last names, respectively.  Use the {\tt cat} command 
      to turn them into a single file {\tt c.txt}.  Do this all within
      the command line and show me the lines you used to accomplish these
      tasks.
\item Suppose, for some reason, you want to execute {\tt a.txt} as a program.
      Show me how you'd set its permissions. 
\end{enumerate} 

\section*{Problem 2 -- Simple Shell Scripting}
 
\begin{enumerate}
\item Write a bash script that converts a temperature from degrees 
      Fahrenheit to degrees Celcius. (Hint: one way is to use {\tt let},
      but be careful about what sorts of numbers that lets you use.)
\item Write a bash script that provides a count of the number of files 
      and subdirectories in the current directory. (Hint: use {\tt grep}.)
\item Write a bash script that takes as input the name of a
      file. The script should move the given file, if it exists, to a 
      directory named trash that is located within 
      your home directory. If the trash directory does not exist, the script 
      should create it. If the given file does not exist, an appropriate 
      error message should be printed. 
\end{enumerate}

\section*{Problem 3 -- Git and Version Control}

\begin{enumerate}
 \item Get a (free) account at GitHub  and create a 
       repository for your shell scripts.  
       Provide me a link to this repository so I can see it.
 \item Modify the temperature conversion script to output the 
       temperature in Kelvin, too, and use 
       use {\tt git} (and GitHub, etc.) to track this change.
\end{enumerate}
 


\end{document}
